\documentclass{report}
\usepackage[utf8]{inputenc}
\usepackage[italian]{babel}
\usepackage[a4paper]{geometry}
\usepackage{hyperref}
\usepackage{graphicx}
\graphicspath{ {./img/} }

\begin{document}

	\centerline{\includegraphics{logo1}}
	\bigbreak

	\begin{center}
		Corso di
        \vskip 0.2cm
		\large{\textbf{Programmazione Logica e Funzionale}}
		\vskip 2cm
		Anno accademico
        \vskip 0.2cm
 		\textbf{2021-2022}
		\vskip 2cm
		Progetto realizzato da
        \vskip 0.2cm
		\textbf{Barzotti Cristian}
        \vskip 0.1cm 290725
        \vskip 0.2cm
        \textbf{Kania Nicholas} 
        \vskip 0.1cm 
        291188
		\vfill
		Corso tenuto dal professore
        \vskip 0.2cm
		\large{\textbf{Bernardo Marco}}
	\end{center}
	
	\part*{Implementazione di DCT e DFT in Haskell e Prolog}
	\tableofcontents % Indice


	\chapter{Specifica del problema}
	Si propone di implementare le funzioni di elaborazione segnali e immagini denominate DCT (\textit{Discrete Cosine Transform}) e DFT (\textit{Discrete Fourier Transform}), e le relative funzioni inverse IDCT e IDFT. \\
	\section*{Nota}
	Ci teniamo a precisare che la funzione DCT è suddivisa in diversi tipi; in questo progetto siamo andati ad implementare quelle che formalmente sono chiamate DCT tipo II e DCT tipo III\footnote{\url{https://en.wikipedia.org/wiki/Discrete_cosine_transform}}.\\ 
	In questa documentazione ci riferiremo alle funzioni rispettivamente come DCT (per la funzione DCT tipo II) e IDCT (per la funzione DCT tipo III).
	
	\chapter{Analisi del problema}
	
	
	\chapter{Progettazione dell'algoritmo}
	
	\chapter{Implementazione dell'algoritmo}

    \chapter{Testing}
    \section{Haskell}
    \section{Prolog}
	
\end{document}